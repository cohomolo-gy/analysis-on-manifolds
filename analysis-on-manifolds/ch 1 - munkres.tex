
\documentclass[10pt, oneside]{article}   	% use "amsart" instead of "article" for AMSLaTeX format
\usepackage{geometry}                		% See geometry.pdf to learn the layout options. There are lots.
\geometry{letterpaper}                   		% ... or a4paper or a5paper or ... 
\usepackage[parfill]{parskip}    		% Activate to begin paragraphs with an empty line rather than an indent
\usepackage{graphicx}				% Use pdf, png, jpg, or eps§ with pdflatex; use eps in DVI mode
								% TeX will automatically convert eps --> pdf in pdflatex		
\usepackage{amssymb}
\usepackage{amsfonts}
\usepackage{amsmath}
\usepackage{amsthm}
\usepackage{bm}
\usepackage{tikz-cd}
\usepackage{enumerate}
\usepackage{xfrac}
\usepackage{hyperref}
\usepackage{unicode-math}
\usepackage{mathtools}

\newcommand{\cat}[1]{\bm{ \mathsf{#1} }}
\newcommand{\functor}[3]{#1 : \cat{#2} \to \cat{#3}}
\newcommand{\functordef}{\functor{F}{C}{D}}
\newcommand{\cc}{\cat{C}}
\newcommand{\dd}{\cat{D}}
\newcommand{\ee}{\cat{E}}
\newcommand{\subcat}[2]{\bm{ \mathsf{#1}}_{\bm{ \mathsf{#2}}}}
\newcommand{\op}[1]{#1^{\text{op}}}
\newcommand{\opc}{\op{\cc}}
\newcommand{\opd}{\op{\dd}}
\newcommand{\ope}{\op{\ee}}
\newcommand{\mono}{\rightarrowtail}
\newcommand{\epi}{\twoheadrightarrow}
\newcommand{\bg}{\cat{BG}}
\newcommand{\bgg}{\cat{BG'}}
\newcommand{\nt}{\Rightarrow}
\newcommand{\ant}[2]{\alpha : F \nt G} 
\newcommand{\bnt}[2]{\beta : F \nt G} 
\newcommand{\anti}[2]{\alpha : F \cong G} 
\newcommand{\bnti}[2]{\beta : F \cong G} 
\newcommand{\zero}{\mathbb{0}}
\newcommand{\one}{\mathbb{1}}
\newcommand{\two}{\mathbb{2}}
\newcommand{\three}{\mathbb{3}}
\newcommand{\innerprod}[2]{\langle #1, #2 \rangle}
\DeclarePairedDelimiter{\norm}{\lVert}{\rVert} 
%SetFonts

\newtheorem{theorem}{Theorem}[section]
\newtheorem{corollary}{Corollary}[theorem]
\newtheorem{lemma}[theorem]{Lemma}
\newtheorem{problem}[theorem]{problem}
%SetFonts

\title{The Algebra and Topology of $\mathbb{R}^n$}
\author{Emily Pillmore}
%\date{}							% Activate to display a given date or no date

\begin{document}
\maketitle
\section{Review of Linear Algebra}

This section was a quick review of linear algebra from Munkres' perspective. It delved into Vector spaces, linear transformations/isomorphisms, rank, transposition, matrices, inner-products and norms.

\begin{problem}[Cauchy-Schwarz Inequality]

Let $V$ be a vector space with inner product $\innerprod{x}{y}$ and norm $\norm{x} = \innerprod{x}{y}^{1/2}$.

\begin{center}
\begin{enumerate}[a)]
	\item Prove the \textbf{Cauchy-Schwarz inequality}: $\innerprod{x}{y} \leq \norm{x}\norm{y}$. \textit{Hint:} If $x,y \neq 0$, set $c = 1/\norm{x}$ and $d = 1/\norm{y}$  and use the fact that $\norm{cx \pm dy} \geq 0$.
	\item Prove that $\norm{x + y} \leq \norm{x} + \norm{y}$. \textit{Hint:} Compute $\innerprod{x + y}{x + y}$ and apply a).
	\item Prove that $\norm{x - y} \geq \norm{x} + \norm{y}$
\end{enumerate}
\end{center}
\end{problem}

\begin{proof} (Cauchy-Schwarz Inequality.)

\par

\begin{enumerate}[a)]
	\item We'll prove this using the hint. Consider the following:
	\begin{equation} 
		\begin{aligned}
			\innerprod{x}{y} &\leq \norm{x}\norm{y} \\
			\innerprod{x}{y}^2 &\leq \innerprod{x}{x}\innerprod{y}{y} \\
			0 &\leq  \innerprod{x}{x} - \frac{\innerprod{x}{y}^2}{\innerprod{y}{y}}
		\end{aligned}
	\end{equation}
	
	Let $\eta = \frac{\innerprod{x}{y}}{\innerprod{y}{y}}$. Then, we have 
	
	\begin{equation}
		\begin{aligned}
			0 &\leq \innerprod{x}{x} - \eta\innerprod{x}{y} \\
			 & \leq \innerprod{x}{x} - \eta\innerprod{y}{x} & \textit{symmetry of the symmetric bilinear form} \\
			 & \leq \innerprod{x - \eta y}{x} \\
		\end{aligned}
	\end{equation}
	Now, what the hint tells us is that if $\norm{ax - by} \geq 0$ then certainly $\innerprod{ax - by}{ax - by} = \norm{ax - by}^2 \geq 0$. So in order to make use of the hint, we need to relate $\innerprod{x - \eta y}{x} $ to something of that form.  We will show that $\innerprod{x - \eta y}{x} = \innerprod{x - \eta y}{x - \eta y}$. 
	
	\begin{equation}
		\begin{aligned}
			 \innerprod{x - \eta y}{x - \eta y}  &=  \innerprod{x - \eta y}{x} - \innerprod{x - \eta y}{- \eta y} \\
			  &= \innerprod{x - \eta y}{x} + \eta \innerprod{x - \eta y}{y} \\
			  &= \innerprod{x - \eta y}{x} + \eta (\innerprod{x}{y} - \eta \innerprod{y}{y}) \\
			  &= \innerprod{x - \eta y}{x} + \eta (\innerprod{x}{y} - \innerprod{x}{y}) \\
			  &=  \innerprod{x - \eta y}{x}
	 	\end{aligned}
	\end{equation}
	Hence, $\innerprod{x - \eta y}{x}$ is equal to $\innerprod{x - \eta y}{x - \eta y}$, which is gauranteed to be $\geq 0$. It is 0 in the case where either $x$ or $y$ is 0, since $\innerprod{x}{y} \leq \norm{x}\norm{y}$ forces the equation to 0. 

\item Consider the following: 

	\begin{equation}
		\begin{aligned}
			\norm{x + y} &\leq \norm{x} + \norm{y} \\
			\innerprod{x + y}{x + y} &\leq (\norm{x} + \norm{y})^2 \\
			\innerprod{x + y}{x + y} &\leq \innerprod{x}{x} + 2\norm{x}\norm{y} + \innerprod{y}{y} \\
			\innerprod{x}{x} + 2\innerprod{x}{y} + \innerprod{y}{y} &\leq \innerprod{x}{x} + 2\norm{x}\norm{y} + \innerprod{y}{y} \\
			\innerprod{x}{y} &\leq \norm{x}\norm{y} 			
		\end{aligned}
	\end{equation}	
	The final clause of the proof is precisely the Cauchy-Schwarz inequality.	
	
	\item Let's begin by considering the following: 
	\begin{equation}
		\begin{aligned}
			\norm{x - y} &\geq \norm{x} - \norm{y} \\
			\norm{x - y}^2 &\geq \innerprod{x}{x} - 2 \norm{x}\norm{y} + \innerprod{y}{y} \\ 
			\innerprod{x - y}{x - y} & \geq \innerprod{x}{x} - 2 \norm{x}\norm{y} + \innerprod{y}{y}  \\
			\innerprod{x}{x} - 2\innerprod{x}{y} + \innerprod{y}{y} &\geq \innerprod{x}{x} - 2 \norm{x}\norm{y} + \innerprod{y}{y}  \\
			\norm{x}\norm{y} &\geq \innerprod{x}{y} 
	 	\end{aligned}
	\end{equation}
	The last clause is again the Cauchy-Schwarz inequality.
\end{enumerate}
\end{proof}

\begin{problem}[Matrix norm]

If $A$ is an $n$ by $m$ matrix, and $B$ is an $m$ by $p$ matrix, show that 

\begin{center}
\begin{equation}
	|A \cdot B| \leq m |A||B|
\end{equation}
\end{center}
\end{problem}

\begin{proof}
Note first that by definition, $A \cdot B = \{ c_{ij} = \sum_{k = 1}^m a_{ik} \cdot b_{kj} \}$. Since each entry $a_{ik} \leq |A|$ and $b_{kj} \leq |B|$ by definition, we have the following: 

\begin{equation}
\begin{aligned}
	|A \cdot B| &= \text{max}\{ |c_{ij}| : c_{ij} \sum_{k = 1}^m a_{ik} \cdot b_{kj} \} \\
	&\leq \text{max}\{ |c_{ij}| : c_{ij} = \sum_{k = 1}^m |A| \cdot |B| \} \\
	&\leq \text{max}\{ |c_{ij}| : c_{ij} = m |A| \cdot |B|\} \\
	&\leq m |A|\cdot|B|
\end{aligned} 
\end{equation}

Hence, $|A \cdot B| \leq m |A| \cdot |B|$.

\end{proof}

\begin{problem}[Sup norm]
Show that the sup norm on $\mathbb{R}^2$ is not derived from the inner product on $\mathbb{R}^2$. Hint: Suppose $\innerprod{x}{y}$ is an inner product on $\mathbb{R}^2$ (\textit{not} the dot product) having the property that $|x| = \innerprod{x}{x}^{1/2}$ . Compute $\innerprod{x \pm y}{x \pm y}$ and apply the case $x = e_1$ and $y = e_2$.
\end{problem}

\begin{proof}

To prove this, we will show a violation of the parallelogram law. Recall that a norm associated with an inner product must satisfy the following law for all $x, y \in \mathbb{R}^2$: 

\begin{center}
\begin{equation}
 \norm{x + y}^2 + \norm{x - y}^2 = 2(\norm{x}^2 + \norm{y}^2)
 \end{equation}
 \end{center}
 
 Consider the sup-norm. Using the parallelogram and plugging in the basis vectors $e_1$ and $e_2$ for $x$ and $y$, we have 
 
 \begin{equation}
 	\begin{aligned}
		| x + y |^2 + | x - y|^2 &= 2(|x|^2 + |y|^2) \\
		| e_1 + e_2 |^2 + | e_1 - e_2|^2 &= 2(|e_1|^2 + |e_2|^2) \\
		| (1,1) |^2 + |(1,-1)^2 &= 2(1 + 1) \\
		1 + 1 &= 2(2) \\
		2 &\neq 4
	\end{aligned}
 \end{equation} 
 A contradiction. Hence, the sup-norm cannot be derived from an inner product.

\end{proof}
\end{document}  